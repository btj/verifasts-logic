\documentclass{article}

\usepackage{xcolor}
\usepackage{amssymb}
\usepackage{mathpartir}

\title{VeriFast's separation logic: a higher-order(ish) logic without laters for modular verification of fine-grained concurrent programs}

\author{Bart Jacobs\\
{\small KU Leuven, Dept. CS, DistriNet Research Group}\\
{\small \textsf{bart.jacobs@cs.kuleuven.be}}}

\definecolor{ghost}{HTML}{CC6600}

\newcommand{\gmapsto}{\mapsto_\mathsf{g}}
\newcommand{\annot}[1]{{\color{blue} #1}}
\newcommand{\ghost}[1]{{\color{ghost} #1}}

\newcommand{\llbrace}{\{\hspace{-3pt}[}
\newcommand{\rrbrace}{]\hspace{-3pt}\}}

\begin{document}

\maketitle

\begin{abstract}
We present the separation logic of our VeriFast tool for modular verification of single-threaded and multithreaded C, Java, and Rust programs. Since the complexities of the programming language are orthogonal to the logic, we present the logic in the context of a simple concurrent programming language.
\end{abstract}

\section{Programming language}

In order to focus on the complexities of the logic rather than those of the programming language, we present VeriFast's separation logic in the context of a trivial concurrent programming language whose syntax is given in Fig.~\ref{fig:program-syntax} and whose small-step operational semantics is given in Fig.~\ref{fig:program-steps}. An example program that allocates a memory cell, increments it twice in parallel, and then asserts that the cell's value equals two is shown in Fig.~\ref{fig:example}.

\begin{figure}
$$\begin{array}{l}
\mathbf{let}\ \mathsf{x} = \mathbf{cons}(0)\ \mathbf{in}\\
(\\
\quad \mathbf{FAA}(\mathsf{x}, 1)\\
||\\
\quad \mathbf{FAA}(\mathsf{x}, 1)\\
);\\
\mathbf{let}\ \mathsf{v} = {*}\mathsf{x}\ \mathbf{in}\\
\mathbf{assert}\ \mathsf{v} = 2
\end{array}$$
\caption{An example program. $\mathbf{cons}(0)$ allocates a memory cell, initializes it to 0, and returns its address. The $\mathbf{FAA}$ command performs a sequentially consistent atomic fetch-and-add operation. $c_1\,||\, c_2$ is the parallel composition of commands $c_1$ and $c_2$. $*\ell$ returns the value stored at address $\ell$.}\label{fig:example}
\end{figure}

\begin{figure}
$$\begin{array}{r @{\;} @{\;} l}
& z \in \mathbb{Z}, x \in \mathcal{X}\\
e ::= & z\ |\ x\\
c ::= & e\ |\ \mathbf{cons}(e)\ |\ \mathbf{FAA}(e, e)\ |\ {*}e\ |\ \mathbf{assert}\ e = e\\
| & \mathbf{let}\ x = c\ \mathbf{in}\ c\ |\ (c\;||\;c)
\end{array}$$
\caption{Syntax of the expressions $e$ and commands $c$ of the programming language. We assume a set $\mathcal{X}$ of program variable names. $c; c'$ is a shorthand for $\mathbf{let}\ \_ = c\ \mathbf{in}\ c'$, where $\_$ is a designated element of $\mathcal{X}$}\label{fig:program-syntax}
\end{figure}

\begin{figure}
\begin{mathpar}
\inferrule{
\ell \notin \mathrm{dom}\,h
}{
(h, \mathbf{cons}(v)) \rightarrow (h[\ell := v], \ell)
}
\and
\inferrule{
\ell \in \mathrm{dom}\,h
}{
(h, \mathbf{FAA}(\ell, z) \rightarrow (h[\ell := h(\ell) + z], h(\ell))
}
\and
\inferrule{
\ell \in \mathrm{dom}\,h
}{
(h, {*}\ell) \rightarrow (h, h(\ell))
}
\and
(h, \mathbf{assert}\ v = v) \rightarrow (h, 0)
\and
(h, \mathbf{let}\ x = v\ \mathbf{in}\ c) \rightarrow (h, c[v/x])
\and
\inferrule{
(h, c) \rightarrow (h', c')
}{
(h, \mathbf{let}\ x = c\ \mathbf{in}\ c'') \rightarrow (h, \mathbf{let}\ x = c'\ \mathbf{in}\ c'')
}
\and
\inferrule{
(h, c) \rightarrow (h', c')
}{
(h, (c\;||\;c'')) \rightarrow (h', (c'\;||\;c''))
}
\and
\inferrule{
(h, c) \rightarrow (h', c')
}{
(h, (c''\;||\;c)) \rightarrow (h', (c''\;||\;c'))
}
\and
(h, v\;||\;v') \rightarrow (h, 0)
\end{mathpar}
\caption{Small-step operational semantics of the programming language}\label{fig:program-steps}
\end{figure}

We define the multiset of threads of a command $c$ as follows:
$$\mathsf{threads}(c) = \left\{\begin{array}{l l}
\mathsf{threads}(c_1) & \textrm{if $c = \mathbf{let}\ x = c_1\ \mathbf{in}\ c_2$}\\
\mathsf{threads}(c_1) \uplus \mathsf{threads}(c_2) & \textrm{if $c = (c_1\;||\;c_2)$}\\
\llbrace c\rrbrace & \textrm{otherwise}
\end{array}\right.$$

We say a configuration $(h, c)$ is \emph{reducible} if it can make a step:
$$\inferrule{
(h, c) \rightarrow (h', c')
}{
\mathsf{red}\,(h, c)
}$$
We say a configuration is \emph{finished} if its command is a value.
$$\mathsf{finished}\,(h, v)$$
We say a configuration is \emph{okay} if each thread is either reducible or finished.
$$\inferrule{
\forall c_\mathsf{t} \in \mathsf{threads}(c).\;\mathsf{finished}\,(h, c_\mathsf{t}) \lor \mathsf{red}\,(h, c_\mathsf{t})
}{
\mathsf{ok}\,(h, c)
}$$
We say a configuration is \emph{safe} if each configuration reachable from it is okay.
$$\inferrule{
\forall h', c'.\;(h, c) \rightarrow^* (h', c') \Rightarrow \mathsf{ok}\,(h', c')
}{
\mathsf{safe}\,(h, c)
}$$
We say a program $c$ is safe if $(\emptyset, c)$ is safe. The goal of the logic we present here is to prove that a given program is safe. This implies that it does not access unallocated memory and that there are no assertion failures.\footnote{In fact, the logic also proves that there are no data races, but for simplicity we do not consider data races here.}

\section{Annotated programs}

\begin{figure}
$$\begin{array}{r @{\;} l r}
& t \in \mathcal{T} & \textrm{lemma type names}\\
& p \in \mathcal{P} & \textrm{predicate constructor names}\\
& g \in \mathcal{G} & \textrm{ghost variable names}\\
& \pi \in \mathbb{R}^+ & \textrm{fractions}\\
\textrm{ghost values $V$} ::= & z\ |\ (V, V)\ |\ ()\ |\ \{\overline{V}\}\\
| & p(\overline{V}) & \textrm{predicate values}\\
| & \lambda \overline{g}.\;C & \textrm{lemma values}\\
\textrm{ghost expressions $E$} ::= & V\ |\ x\ |\ g\ |\ E + E\\
| & p(\overline{E}) & \textrm{predicate constructor applications}\\
| & (E, E)\ |\ () & \textrm{pair expressions, empty tuple}\\
| & \emptyset\ |\ \{E\}\ |\ E \cup E & \textrm{set expressions}\\
\textrm{assertions $a$} ::= & [\pi]E \mapsto E & \textrm{points-to assertions}\\
| & [\pi]E \gmapsto E & \textrm{ghost cell points-to assertions}\\
| & E() & \textrm{predicate assertions}\\
| & [\pi]\mathbf{atomic\_space}(E, E) & \textrm{atomic space assertions}\\
| & E : t(\overline{E}) & \textrm{lemma type assertions}\\
| & \exists g.\;a\\
| & \mathbf{atomic\_spaces}(E) & \textrm{atomic spaces assertions}\\
\mathit{gdecl} ::= & \multicolumn{2}{@{} l @{}}{\mathbf{lem\_type}\ t(\overline{g}) = \mathbf{lem}(\overline{g})\ \mathbf{forall}\ \overline{g}\ \mathbf{req}\ a\ \mathbf{ens}\ a}\\
| & \mathbf{pred\_ctor}\ p(\overline{g})() = a\\
C ::= & \multicolumn{2}{@{} l @{}}{c\ |\ \mathbf{let}\ x = c\ \mathbf{in}\ C\ |\ \mathbf{glet}\ g = C\ \mathbf{in}\ C\ |\ E(\overline{E})}\\
| & \mathbf{gcons}(E)\ |\ {*}E \leftarrow_\mathsf{g} E\\
| & \mathbf{open}\ E()\ |\ \mathbf{close}\ E()\\
| & \multicolumn{2}{@{} l @{}}{\mathbf{create\_atomic\_space}(E, E)\ |\ \mathbf{destroy\_atomic\_space}(E, E)}\\
| & \multicolumn{2}{@{} l @{}}{\mathbf{open\_atomic\_space}(E, E)\ |\ \mathbf{close\_atomic\_space}(E, E)}\\
| & \multicolumn{2}{@{} l @{}}{\mathbf{produce\_lem\_ptr\_chunk}\ t(\overline{E})(\overline{g})\ \{\ C\ \}}\\
\end{array}$$
\caption{Syntax of ghost declarations $\mathit{gdecl}$ and annotated commands $C$}
\end{figure}

\begin{figure}
$$\ghost{\begin{array}{@{} l @{}}
\mathbf{lem\_type}\ \mathsf{FAA\_op}(\mathsf{x}, \mathsf{n}, \mathsf{P}, \mathsf{Q}) = \mathbf{lem}()\\
\quad \mathbf{forall}\ \mathsf{v}\\
\quad \mathbf{req}\ \mathsf{x} \mapsto \mathsf{v} * \mathsf{P}()\\
\quad \mathbf{ens}\ \mathsf{x} \mapsto \mathsf{v} + \mathsf{n} * \mathsf{Q}()\\
\mathbf{lem\_type}\ \mathsf{FAA\_ghop}(\mathsf{x}, \mathsf{n}, \mathsf{pre}, \mathsf{post}) = \mathbf{lem}(\mathsf{op})\\
\quad \mathbf{forall}\ \mathsf{P}, \mathsf{Q}\\
\quad \mathbf{req}\ \mathbf{atomic\_spaces}(\emptyset) * \mathsf{op} : \mathsf{FAA\_op}(\mathsf{x}, \mathsf{n}, \mathsf{P}, \mathsf{Q}) * \mathsf{P}() * \mathsf{pre}()\\
\quad \mathbf{ens}\ \mathbf{atomic\_spaces}(\emptyset) * \mathsf{op} : \mathsf{FAA\_op}(\mathsf{x}, \mathsf{n}, \mathsf{P}, \mathsf{Q}) * \mathsf{Q}() * \mathsf{post}()\\
\end{array}}$$
\caption{The ghost prelude (built-in ghost declarations)}\label{fig:prelude}
\end{figure}

\begin{figure}
$$\begin{array}{l}
\ghost{\begin{array}{@{} l @{}}
\mathbf{pred\_ctor}\ \mathsf{Inv}(\mathsf{x}, \mathsf{g1}, \mathsf{g2})() = \exists \mathsf{v1}, \mathsf{v2}.\;[1/2]\mathsf{g1} \gmapsto \mathsf{v1} * [1/2]\mathsf{g2} \gmapsto \mathsf{v2} * \mathsf{x} \mapsto \mathsf{v1} + \mathsf{v2}\\
\mathbf{pred\_ctor}\ \mathsf{pre1}(\mathsf{x}, \mathsf{g1}, \mathsf{g2})() = [1/2]\mathbf{atomic\_space}(\mathsf{Nx}, \mathsf{Inv}(\mathsf{x}, \mathsf{g1}, \mathsf{g2})) * [1/2]\mathsf{g1} \gmapsto 0\\
\mathbf{pred\_ctor}\ \mathsf{post1}(\mathsf{x}, \mathsf{g1}, \mathsf{g2})() = [1/2]\mathbf{atomic\_space}(\mathsf{Nx}, \mathsf{Inv}(\mathsf{x}, \mathsf{g1}, \mathsf{g2})) * [1/2]\mathsf{g1} \gmapsto 1\\
\mathbf{pred\_ctor}\ \mathsf{pre2}(\mathsf{x}, \mathsf{g1}, \mathsf{g2})() = [1/2]\mathbf{atomic\_space}(\mathsf{Nx}, \mathsf{Inv}(\mathsf{x}, \mathsf{g1}, \mathsf{g2})) * [1/2]\mathsf{g2} \gmapsto 0\\
\mathbf{pred\_ctor}\ \mathsf{post2}(\mathsf{x}, \mathsf{g1}, \mathsf{g2})() = [1/2]\mathbf{atomic\_space}(\mathsf{Nx}, \mathsf{Inv}(\mathsf{x}, \mathsf{g1}, \mathsf{g2})) * [1/2]\mathsf{g2} \gmapsto 1
\end{array}}\\
\\
\mathbf{let}\ \mathsf{x} = \mathbf{cons}(0)\ \mathbf{in}\\
\ghost{\mathbf{glet}\ \mathsf{g1} = \mathbf{gcons}(0)\ \mathbf{in}}\\
\ghost{\mathbf{glet}\ \mathsf{g2} = \mathbf{gcons}(0)\ \mathbf{in}}\\
\ghost{\mathbf{close}\ \mathsf{Inv}(\mathsf{x}, \mathsf{g1}, \mathsf{g2})();}\\
\ghost{\mathbf{create\_atomic\_space}(\mathsf{Nx}, \mathsf{Inv}(\mathsf{x}, \mathsf{g1}, \mathsf{g2}));}\\
(\\
\quad \ghost{\begin{array}{@{} l @{}}
\mathbf{produce\_lem\_ptr\_chunk}\ \mathsf{FAA\_ghop}(\mathsf{x}, 1, \mathsf{pre1}(\mathsf{x}, \mathsf{g1}, \mathsf{g2}), \mathsf{post1}(\mathsf{x}, \mathsf{g1}, \mathsf{g2}))(\mathsf{op})\ \{\\
\quad \mathbf{open}\ \mathsf{pre1}(\mathsf{x}, \mathsf{g1}, \mathsf{g2})();\\
\quad \mathbf{open\_atomic\_space}(\mathsf{Nx}, \mathsf{Inv}(\mathsf{x}, \mathsf{g1}, \mathsf{g2}));\ \mathbf{open}\ \mathsf{Inv}(\mathsf{x}, \mathsf{g1}, \mathsf{g2})();\\
\quad \mathsf{op}();\\
\quad *\mathsf{g1} \leftarrow_\mathsf{g} 1;\\
\quad \mathbf{close}\ \mathsf{Inv}(\mathsf{x}, \mathsf{g1}, \mathsf{g2})();\ \mathbf{close\_atomic\_space}(\mathsf{Nx}, \mathsf{Inv}(\mathsf{x}, \mathsf{g1}, \mathsf{g2}));\\
\quad \mathbf{close}\ \mathsf{post1}(\mathsf{x}, \mathsf{g1}, \mathsf{g2})()\\
\};
\end{array}}\\
\quad \ghost{\mathbf{close}\ \mathsf{pre1}(\mathsf{x}, \mathsf{g1}, \mathsf{g2})();}\\
\quad \mathbf{FAA}(\mathsf{x}, 1);\\
\quad \ghost{\mathbf{open}\ \mathsf{post1}(\mathsf{x}, \mathsf{g1}, \mathsf{g2})()}\\
||\\
\quad \ghost{\begin{array}{@{} l @{}}
\mathbf{produce\_lem\_ptr\_chunk}\ \mathsf{FAA\_ghop}(\mathsf{x}, 1, \mathsf{pre2}(\mathsf{x}, \mathsf{g1}, \mathsf{g2}), \mathsf{post2}(\mathsf{x}, \mathsf{g1}, \mathsf{g2}))(\mathsf{op})\ \{\\
\quad \mathbf{open}\ \mathsf{pre2}(\mathsf{x}, \mathsf{g1}, \mathsf{g2})();\\
\quad \mathbf{open\_atomic\_space}(\mathsf{Nx}, \mathsf{Inv}(\mathsf{x}, \mathsf{g1}, \mathsf{g2}));\ \mathbf{open}\ \mathsf{Inv}(\mathsf{x}, \mathsf{g1}, \mathsf{g2})();\\
\quad \mathsf{op}();\\
\quad *\mathsf{g2} \leftarrow_\mathsf{g} 1;\\
\quad \mathbf{close}\ \mathsf{Inv}(\mathsf{x}, \mathsf{g1}, \mathsf{g2})();\ \mathbf{close\_atomic\_space}(\mathsf{Nx}, \mathsf{Inv}(\mathsf{x}, \mathsf{g1}, \mathsf{g2}));\\
\quad \mathbf{close}\ \mathsf{post2}(\mathsf{x}, \mathsf{g1}, \mathsf{g2})()\\
\};
\end{array}}\\
\quad \ghost{\mathbf{close}\ \mathsf{pre2}(\mathsf{x}, \mathsf{g1}, \mathsf{g2})();}\\
\quad \mathbf{FAA}(\mathsf{x}, 1);\\
\quad \ghost{\mathbf{open}\ \mathsf{post2}(\mathsf{x}, \mathsf{g1}, \mathsf{g2})()}\\
);\\
\ghost{\mathbf{destroy\_atomic\_space}(\mathsf{Nx}, \mathsf{Inv}(\mathsf{x}, \mathsf{g1}, \mathsf{g2}));}\\
\ghost{\mathbf{open}\ \mathsf{Inv}(\mathsf{x}, \mathsf{g1}, \mathsf{g2})();}\\
\mathbf{let}\ \mathsf{v} = {*}\mathsf{x}\ \mathbf{in}\\
\mathbf{assert}\ \mathsf{v} = 2
\end{array}$$
\caption{VeriFast proof of the example program. $\mathsf{Nx} \triangleq ()$.}\label{fig:example-proof}
\end{figure}

\begin{figure}
$$\begin{array}{l}
\annot{\{\mathbf{emp}\}}\\
\mathbf{let}\ \mathsf{x} = \mathbf{cons}(0)\ \mathbf{in}\ \ghost{\mathbf{glet}\ \mathsf{g1} = \mathbf{gcons}(0)\ \mathbf{in}}\ \ghost{\mathbf{glet}\ \mathsf{g2} = \mathbf{gcons}(0)\ \mathbf{in}}\\
\annot{\{\mathsf{x} \mapsto 0 * \mathsf{g1} \gmapsto 0 * \mathsf{g2} \gmapsto 0\}}\\
\ghost{\mathbf{close}\ \mathsf{Inv}(\mathsf{x}, \mathsf{g1}, \mathsf{g2})();}\\
\annot{\{\mathsf{Inv}(\mathsf{x}, \mathsf{g1}, \mathsf{g2})() * [1/2]\mathsf{g1} \gmapsto 0 * [1/2]\mathsf{g2} \gmapsto 0\}}\\
\ghost{\mathbf{create\_atomic\_space}(\mathsf{Nx}, \mathsf{Inv}(\mathsf{x}, \mathsf{g1}, \mathsf{g2}));}\\
\annot{\{\mathbf{atomic\_space}(\mathsf{Inv}(\mathsf{x}, \mathsf{g1}, \mathsf{g2})) * [1/2]\mathsf{g1} \gmapsto 0 * [1/2]\mathsf{g2} \gmapsto 0\}}\\
(\\
\quad \annot{\{[1/2]\mathbf{atomic\_space}(\mathsf{Inv}(\mathsf{x}, \mathsf{g1}, \mathsf{g2})) * [1/2]\mathsf{g1} \gmapsto 0\}}\\
\quad \ghost{\begin{array}{@{} l @{}}
\mathbf{glet}\ \mathsf{lem} = \mathbf{produce\_lem\_ptr\_chunk}\ \mathsf{FAA\_ghop}(\mathsf{x}, 1, \mathsf{pre1}(\mathsf{x}, \mathsf{g1}, \mathsf{g2}), \mathsf{post1}(\mathsf{x}, \mathsf{g1}, \mathsf{g2}))(\mathsf{op})\ \{\\
\quad \annot{\textrm{For all $\mathsf{P}, \mathsf{Q},$}}\\
\quad \annot{\{\mathbf{atomic\_spaces}(\emptyset) * \mathsf{op} : \mathsf{FAA\_op}(\mathsf{x}, 1, \mathsf{P}, \mathsf{Q}) * \mathsf{P}() * \mathsf{pre1}(\mathsf{x}, \mathsf{g1}, \mathsf{g2})()\}}\\
\quad \mathbf{open}\ \mathsf{pre1}(\mathsf{x}, \mathsf{g1}, \mathsf{g2})();\\
\quad \annot{\left\{\begin{array}{l}
\mathbf{atomic\_spaces}(\emptyset) * \mathsf{op} : \mathsf{FAA\_op}(\mathsf{x}, 1, \mathsf{P}, \mathsf{Q}) * \mathsf{P}() * {}\\
{}[1/2]\mathbf{atomic\_space}(\mathsf{Nx}, \mathsf{Inv}(\mathsf{x}, \mathsf{g1}, \mathsf{g2})) * [1/2]\mathsf{g1} \gmapsto 0
\end{array}\right\}}\\
\quad \mathbf{open\_atomic\_space}(\mathsf{Nx}, \mathsf{Inv}(\mathsf{x}, \mathsf{g1}, \mathsf{g2}));\ \mathbf{open}\ \mathsf{Inv}(\mathsf{x}, \mathsf{g1}, \mathsf{g2})();\\
\quad \annot{\left\{\begin{array}{l}
\exists \mathsf{v2}.\;\mathbf{atomic\_spaces}(\{(\mathsf{Nx}, \mathsf{Inv}(\mathsf{x}, \mathsf{g1}, \mathsf{g2})\}) * \mathsf{op} : \mathsf{FAA\_op}(\mathsf{x}, 1, \mathsf{P}, \mathsf{Q}) * \mathsf{P}() * {}\\
{}[1/2]\mathbf{atomic\_space}(\mathsf{Nx}, \mathsf{Inv}(\mathsf{x}, \mathsf{g1}, \mathsf{g2})) * \mathsf{g1} \gmapsto 0 * [1/2]\mathsf{g2} \gmapsto \mathsf{v2} * \mathsf{x} \mapsto \mathsf{v2}
\end{array}\right\}}\\
\quad \annot{\textrm{For all $\mathsf{v2}$,}}\\
\quad \annot{\left\{\begin{array}{l}
\mathbf{atomic\_spaces}(\{(\mathsf{Nx}, \mathsf{Inv}(\mathsf{x}, \mathsf{g1}, \mathsf{g2})\}) * \mathsf{op} : \mathsf{FAA\_op}(\mathsf{x}, 1, \mathsf{P}, \mathsf{Q}) * \mathsf{P}() * {}\\
{}[1/2]\mathbf{atomic\_space}(\mathsf{Nx}, \mathsf{Inv}(\mathsf{x}, \mathsf{g1}, \mathsf{g2})) * \mathsf{g1} \gmapsto 0 * [1/2]\mathsf{g2} \gmapsto \mathsf{v2} * \mathsf{x} \mapsto \mathsf{v2}
\end{array}\right\}}\\
\quad \mathsf{op}();\\
\quad \annot{\left\{\begin{array}{l}
\mathbf{atomic\_spaces}(\{(\mathsf{Nx}, \mathsf{Inv}(\mathsf{x}, \mathsf{g1}, \mathsf{g2})\}) * \mathsf{op} : \mathsf{FAA\_op}(\mathsf{x}, 1, \mathsf{P}, \mathsf{Q}) * \mathsf{Q}() * {}\\
{}[1/2]\mathbf{atomic\_space}(\mathsf{Nx}, \mathsf{Inv}(\mathsf{x}, \mathsf{g1}, \mathsf{g2})) * \mathsf{g1} \gmapsto 0 * [1/2]\mathsf{g2} \gmapsto \mathsf{v2} * \mathsf{x} \mapsto 1 + \mathsf{v2}
\end{array}\right\}}\\
\quad *\mathsf{g1} \leftarrow_\mathsf{g} 1;\\
\quad \annot{\left\{\begin{array}{l}
\mathbf{atomic\_spaces}(\{(\mathsf{Nx}, \mathsf{Inv}(\mathsf{x}, \mathsf{g1}, \mathsf{g2})\}) * \mathsf{op} : \mathsf{FAA\_op}(\mathsf{x}, 1, \mathsf{P}, \mathsf{Q}) * \mathsf{Q}() * {}\\
{}[1/2]\mathbf{atomic\_space}(\mathsf{Nx}, \mathsf{Inv}(\mathsf{x}, \mathsf{g1}, \mathsf{g2})) * \mathsf{g1} \gmapsto 1 * [1/2]\mathsf{g2} \gmapsto \mathsf{v2} * \mathsf{x} \mapsto 1 + \mathsf{v2}
\end{array}\right\}}\\
\quad \mathbf{close}\ \mathsf{Inv}(\mathsf{x}, \mathsf{g1}, \mathsf{g2})();\ \mathbf{close\_atomic\_space}(\mathsf{Nx}, \mathsf{Inv}(\mathsf{x}, \mathsf{g1}, \mathsf{g2}));\\
\quad \annot{\left\{\begin{array}{l}
\mathbf{atomic\_spaces}(\emptyset) * \mathsf{op} : \mathsf{FAA\_op}(\mathsf{x}, 1, \mathsf{P}, \mathsf{Q}) * \mathsf{Q}() * {}\\
{}[1/2]\mathbf{atomic\_space}(\mathsf{Nx}, \mathsf{Inv}(\mathsf{x}, \mathsf{g1}, \mathsf{g2})) * [1/2]\mathsf{g1} \gmapsto 1
\end{array}\right\}}\\
\quad \mathbf{close}\ \mathsf{post1}(\mathsf{x}, \mathsf{g1}, \mathsf{g2})()\\
\quad \annot{\{\mathbf{atomic\_spaces}(\emptyset) * \mathsf{op} : \mathsf{FAA\_op}(\mathsf{x}, 1, \mathsf{P}, \mathsf{Q}) * \mathsf{Q}() * \mathsf{post1}(\mathsf{x}, \mathsf{g1}, \mathsf{g2})()\}}\\
\};
\end{array}}\\
\quad \annot{\{[1/2]\mathbf{atomic\_space}(\mathsf{Inv}(\mathsf{x}, \mathsf{g1}, \mathsf{g2})) * [1/2]\mathsf{g1} \gmapsto 0 * \mathsf{lem} : \mathsf{FAA\_ghop}(\mathsf{x}, 1, \mathsf{pre1}(\mathsf{x}, \mathsf{g1}, \mathsf{g2}), \mathsf{post1}(\mathsf{x}, \mathsf{g1}, \mathsf{g2}))\}}\\
\quad \ghost{\mathbf{close}\ \mathsf{pre1}(\mathsf{x}, \mathsf{g1}, \mathsf{g2})();}\\
\quad \annot{\{\mathsf{pre1}(\mathsf{x}, \mathsf{g1}, \mathsf{g2})() * \mathsf{lem} : \mathsf{FAA\_ghop}(\mathsf{x}, 1, \mathsf{pre1}(\mathsf{x}, \mathsf{g1}, \mathsf{g2}), \mathsf{post1}(\mathsf{x}, \mathsf{g1}, \mathsf{g2}))\}}\\
\quad \mathbf{FAA}(\mathsf{x}, 1);\\
\quad \annot{\{\mathsf{post1}(\mathsf{x}, \mathsf{g1}, \mathsf{g2})() * \mathsf{lem} : \mathsf{FAA\_ghop}(\mathsf{x}, 1, \mathsf{pre1}(\mathsf{x}, \mathsf{g1}, \mathsf{g2}), \mathsf{post1}(\mathsf{x}, \mathsf{g1}, \mathsf{g2}))\}}\\
\quad \ghost{\mathbf{open}\ \mathsf{post1}(\mathsf{x}, \mathsf{g1}, \mathsf{g2})()}\\
\quad \annot{\{[1/2]\mathbf{atomic\_space}(\mathsf{Inv}(\mathsf{x}, \mathsf{g1}, \mathsf{g2})) * [1/2]\mathsf{g1} \gmapsto 1\}}\\
||\\
\quad \dots\\
);\\
\annot{\{\mathbf{atomic\_space}(\mathsf{Inv}(\mathsf{x}, \mathsf{g1}, \mathsf{g2})) * [1/2]\mathsf{g1} \gmapsto 1 * [1/2]\mathsf{g2} \gmapsto 1\}}\\
\ghost{\mathbf{destroy\_atomic\_space}(\mathsf{Nx}, \mathsf{Inv}(\mathsf{x}, \mathsf{g1}, \mathsf{g2}));}\ \ghost{\mathbf{open}\ \mathsf{Inv}(\mathsf{x}, \mathsf{g1}, \mathsf{g2})();}\\
\annot{\{\mathsf{g1} \gmapsto 1 * \mathsf{g2} \gmapsto 1 * \mathsf{x} \mapsto 2\}}\\
\mathbf{let}\ \mathsf{v} = {*}\mathsf{x}\ \mathbf{in}\\
\mathbf{assert}\ \mathsf{v} = 2
\end{array}$$
\caption{Proof outline for the example proof}\label{fig:example-outline}
\end{figure}

\end{document}
